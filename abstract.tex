\documentclass{article}
\usepackage[utf8]{inputenc}

\title{Cloud Atlas}
\subtitle{An LstmEncoder for UHECR AirShowers}
\author{Gianluca Becuzzi, Lucia Papalini}
\date{July 2022}

\begin{document}

\maketitle

In this project we propose an example of using deep learning techniques to predict features of UHECR 
(Ultra High Energy Cosmic Rays) Showers.
In particular when this kind of cosmic ray enters the atmosphere they produce a particle cascade 
that can be identified with a set of water-Cherenkov or scintillator ground-based detectors.\\
The dataset on which we perform the training consists in a simulation of an Auger-like apparatus, 
made of a 9x9 array of detectors.  Data are organised organised as follow:
\begin{itemize}  % @djanloo
    \item[-]\textit{time of arrival}
    \item[-]\textit{time series}
    \item[-]\textit{outcome}
\end{itemize}



\end{document}